\documentclass[a4paper]{article}

\usepackage[margin=1.5cm]{geometry}
\usepackage{amsmath,amsthm,amssymb,tabu}
\usepackage[spanish,es-tabla]{babel}
\decimalpoint
\usepackage[T1]{fontenc}
\usepackage[utf8]{inputenc}
\usepackage{lmodern}
\usepackage[dvipsnames]{xcolor}
\usepackage[hyphens]{url}
\usepackage{graphicx}
\graphicspath{ {images/} }
\usepackage{tcolorbox}
\usepackage{enumitem}
\setcounter{section}{-1}
\usepackage{tabularx}
\usepackage{multirow}
\usepackage{hyperref}
\usepackage{braket}
\usepackage{tikz}
\usepackage{mathrsfs}
\usetikzlibrary{cd}
\usepackage{pgfplots}
\usepackage{caption}
\usepackage{subcaption}
\usetikzlibrary{babel}

\definecolor{NARANJA}{rgb}{1,0.467,0}
\definecolor{VERDE}{rgb}{0.31,1,0}
\definecolor{AZUL}{rgb}{0,0.53,1}
\definecolor{ROJO}{rgb}{1,0,0}


\hypersetup{
    colorlinks=true,
    linkcolor=ROJO,
    filecolor=magenta,
    urlcolor=AZUL,
}
\newenvironment{theorem}[2][Theorem]{\begin{trivlist}
\item[\hskip \labelsep {\bfseries #1}\hskip \labelsep {\bfseries #2.}]}{\end{trivlist}}
\newenvironment{teorema}[2][Teorema]{\begin{trivlist}
\item[\hskip \labelsep {\bfseries #1}\hskip \labelsep {\bfseries #2.}]}{\end{trivlist}}
\newenvironment{lema}[2][Lema]{\begin{trivlist}
\item[\hskip \labelsep {\bfseries #1}\hskip \labelsep {\bfseries #2.}]}{\end{trivlist}}
\newenvironment{exercise}[2][Exercise]{\begin{trivlist}

\item[\hskip \labelsep {\bfseries #1}\hskip \labelsep {\bfseries #2.}]}{\end{trivlist}}
\newenvironment{problem}[2][Problem]{\begin{trivlist}
\item[\hskip \labelsep {\bfseries #1}\hskip \labelsep {\bfseries #2.}]}{\end{trivlist}}
\newenvironment{question}[2][Question]{\begin{trivlist}
\item[\hskip \labelsep {\bfseries #1}\hskip \labelsep {\bfseries #2.}]}{\end{trivlist}}
\newenvironment{corollary}[2][Corollary]{\begin{trivlist}
\item[\hskip \labelsep {\bfseries #1}\hskip \labelsep {\bfseries #2.}]}{\end{trivlist}}
\newenvironment{corolario}[2][Corolario]{\begin{trivlist}
\item[\hskip \labelsep {\bfseries #1}]}{\end{trivlist}}
\newenvironment{solution}{\begin{proof}[Solution]}{\end{proof}}

\pgfplotsset{compat=1.15}

\begin{document}
\title{Temario de Algoritmos Computacionales}
\author{Diego Alberto Barceló Nieves \\ Facultad de Ciencias \\ Universidad Nacional Autónoma de México}
\date{}
\maketitle

El siguiente temario está basado en el \href{https://www.fciencias.unam.mx/estudiar-en-ciencias/estudios/licenciaturas/asignaturas/2016/1430}{temario oficial del curso}. Al elaborarlo, asumimos que quienes tomarán el curso no tienen experiencia previa con programación, pero sí tienen bases teóricas sólidas de álgebra superior y cálculo diferencial e integral de una variable, así como nociones básicas de ecuaciones diferenciales ordinarias, pues esto último será necesario en el módulo 3. Para los módulos 1, 2 y 3 utilizaremos el lenguaje de programación \href{https://julialang.org/}{Julia}. La duración aproximada de cada módulo se indica en paréntesis.

\setcounter{section}{-1}

\section{Introducción a la programación (2 semanas)} \label{Sec: Introducción a la programación (2 semanas)} 

\begin{enumerate}[label=\arabic*.]

    \item ¿Qué es un programa? (Paradigma imperativo de la programación)

    \item ¿Cómo se ejecuta un programa? (Lenguaje de programación, código fuente y sintáxis, comentarios y mensajes de error)
    \item Licencias: legalidad y ética. (Software privativo, software de código abierto y software libre)
    \item ¿Cómo escribo y corro un programa? (Editor de texto y terminal virtual, REPLs e IDEs)
    \item ¿Cómo aprendo a programar? (Manuales, documentación y foros de preguntas)
    \item Herramientas útiles para hacer programación. (\href{https://jupyter.org/}{Jupyter} y \href{https://github.com/fonsp/Pluto.jl/blob/main/README.md}{Pluto}, \href{https://git-scm.com/}{Git} y \href{https://github.com/}{GitHub}/\href{https://about.gitlab.com/}{GitLab)}
\end{enumerate}

\section{Estructura básica de la programación (6 semanas)} \label{Sec: Estructura básica de la programación (3 semanas)}

\begin{enumerate}[label=\arabic*.]

    \item Operadores aritméticos y tipos de datos numéricos. (Precedencia y asociatividad de operadores)
    \item Sistemas numéricos de punto flotante y error numérico. (Épsilon de máquina y propagación de errores)
    \item Estructura lógica. (Valores Booleanos, operadores lógicos y operadores de comparación)
    \item Condicionales y ciclos. (Declaraciones \emph{if}, \emph{else} y \emph{elseif}, ciclos \emph{for} y \emph{while})
    \item Conceptos fundamentales. (Arreglos, variables, constantes y funciones)
    \item Algoritmos y diagramas de flujo.
    \item Métodos numéricos. (Estabilidad y convergencia).
    \item Estructura de la programación modular. (Bibliotecas)
\end{enumerate}

\section{Representaciones visuales (2 semanas)} \label{Sec: Representaciones visuales (2 semanas)}

\begin{enumerate}[label=\arabic*.]

    \item Gráficación de funciones, animaciones y visualización de datos  con \href{https://docs.juliaplots.org/latest/}{Plots}.
    \item Manipulación de imágenes digitales con \href{https://juliaimages.org/latest/}{JuliaImages}.
\end{enumerate}

\section{Cómputo científico (5 semanas)} \label{Sec: Cómputo científico: construcción de pseudocódigo e implementación en código (8 semanas)}

\begin{enumerate}[label=\arabic*.]

    \item Solución de sistemas lineales de ecuaciones algebráicas. (Método de eliminación Gaussiana)
    \item Aproximación de raíces. (Método de Newton)
    \item Solución de ecuaciones diferenciales ordinarias. (Método de Euler)
    \item Caminante aleatorio.
\end{enumerate}

\section{Introducción a otros lenguajes de programación (1 semana)} \label{Sec: Introducción a otros lenguajes de programación (1 semana)}

\begin{enumerate}[label=\arabic*.]

    \item GNU Octave.
    \item R.
\end{enumerate}

\section*{Bibliografía recomandada para el curso} \label{Sec: Bibliografía}

\begin{enumerate}

    \item \href{https://benlauwens.github.io/ThinkJulia.jl/latest/book.html}{Lauwens y Downey, \emph{Think Julia: How to Think Like a Computer Scientist} (2019).}

    \item Burden, Faires y Burden, \emph{Numerical Analysis} (2016).

    \item Cormen, Leiserson, Rivest y Stein, \emph{Introduction to Algorithms} (2009).

    \item Cairó, \emph{Metodología de la programacióñ: Algoritmos, diagramas de flujo y programas} (2003).
\end{enumerate}

\section*{Bibliografía complementaria} \label{Sec: Bibliografía complementaria}

\begin{enumerate}

    \item \href{https://docs.julialang.org/en/v1/}{Documentación de Julia}.

    \item \href{https://docs.jupyter.org/en/latest/}{Documentación de Jupyter}.

    \item Material del curso \href{https://computationalthinking.mit.edu/Spring21/}{``Introduction to Computational Thinking''} del MIT.

    \item Blum y Bresnahan - \emph{Linux Command Line and Shell Scripting Bible} (2015).
\end{enumerate}


\end{document}
