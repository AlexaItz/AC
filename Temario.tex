\documentclass[a4paper]{article}

\usepackage[margin=1.5cm]{geometry}
\usepackage{amsmath,amsthm,amssymb,tabu}
\usepackage[spanish,es-tabla]{babel}
\decimalpoint
\usepackage[T1]{fontenc}
\usepackage[utf8]{inputenc}
\usepackage{lmodern}
\usepackage[dvipsnames]{xcolor}
\usepackage[hyphens]{url}
\usepackage{graphicx}
\graphicspath{ {images/} }
\usepackage{tcolorbox}
\usepackage{enumitem}
\setcounter{section}{-1}
\usepackage{tabularx}
\usepackage{multirow}
\usepackage{hyperref}
\usepackage{braket}
\usepackage{tikz}
\usepackage{mathrsfs}
\usetikzlibrary{cd}
\usepackage{pgfplots}
\usepackage{caption}
\usepackage{subcaption}
\usetikzlibrary{babel}

\definecolor{NARANJA}{rgb}{1,0.467,0}
\definecolor{VERDE}{rgb}{0.31,1,0}
\definecolor{AZUL}{rgb}{0,0.53,1}
\definecolor{ROJO}{rgb}{1,0,0}


\hypersetup{
    colorlinks=true,
    linkcolor=ROJO,
    filecolor=magenta,
    urlcolor=AZUL,
}
\newenvironment{theorem}[2][Theorem]{\begin{trivlist}
\item[\hskip \labelsep {\bfseries #1}\hskip \labelsep {\bfseries #2.}]}{\end{trivlist}}
\newenvironment{teorema}[2][Teorema]{\begin{trivlist}
\item[\hskip \labelsep {\bfseries #1}\hskip \labelsep {\bfseries #2.}]}{\end{trivlist}}
\newenvironment{lema}[2][Lema]{\begin{trivlist}
\item[\hskip \labelsep {\bfseries #1}\hskip \labelsep {\bfseries #2.}]}{\end{trivlist}}
\newenvironment{exercise}[2][Exercise]{\begin{trivlist}

\item[\hskip \labelsep {\bfseries #1}\hskip \labelsep {\bfseries #2.}]}{\end{trivlist}}
\newenvironment{problem}[2][Problem]{\begin{trivlist}
\item[\hskip \labelsep {\bfseries #1}\hskip \labelsep {\bfseries #2.}]}{\end{trivlist}}
\newenvironment{question}[2][Question]{\begin{trivlist}
\item[\hskip \labelsep {\bfseries #1}\hskip \labelsep {\bfseries #2.}]}{\end{trivlist}}
\newenvironment{corollary}[2][Corollary]{\begin{trivlist}
\item[\hskip \labelsep {\bfseries #1}\hskip \labelsep {\bfseries #2.}]}{\end{trivlist}}
\newenvironment{corolario}[2][Corolario]{\begin{trivlist}
\item[\hskip \labelsep {\bfseries #1}]}{\end{trivlist}}
\newenvironment{solution}{\begin{proof}[Solution]}{\end{proof}}

\pgfplotsset{compat=1.15}

\begin{document}
\title{Temario de Algoritmos Computacionales}
\author{M. en C. Diego Alberto Barceló Nieves \\ Facultad de Ciencias \\ Universidad Nacional Autónoma de México}
\date{}
\maketitle

El siguiente temario está basado en el \href{https://www.fciencias.unam.mx/estudiar-en-ciencias/estudios/licenciaturas/asignaturas/2016/1430}{temario oficial del curso}. Al elaborarlo, asumimos que quienes tomarán el curso no tienen experiencia previa con programación, pero sí tienen bases teóricas sólidas de álgebra lineal y cálculo diferencial e integral de una variable, así como nociones básicas de ecuaciones diferenciales ordinarias. Para los Módulos 1, 2 y 3 utilizaremos el lenguaje de programación \href{https://julialang.org/}{Julia}. La duración aproximada de cada módulo se indica en paréntesis.

\setcounter{section}{-1}

\section{Introducción a la programación (3 semanas)} \label{Sec: Introducción a la programación (3 semanas)} 

\begin{enumerate}[label=\arabic*.]

    \item ¿Qué es un programa? (Paradigma imperativo de la programación)

    \item ¿Qué ocurre cuando se ejecuta un programa? (Código de máquina y código fuente, lenguajes de programación, sintáxis y semántica, comentarios y mensajes de error)
    \item Licencias: legalidad y ética. (\emph{Software} de código cerrado y de código abierto, \emph{software} privativo y \emph{software} libre)
    \item ¿Cómo escribo y ejecuto un programa? (Editor de texto y terminal virtual, REPLs e IDEs)
    \item ¿Cómo aprendo a programar? (Manuales, documentación y foros de preguntas)
    \item Herramientas útiles para hacer programación. (\href{https://jupyter.org/}{Jupyter} y \href{https://github.com/fonsp/Pluto.jl/blob/main/README.md}{Pluto}, \href{https://git-scm.com/}{Git} y \href{https://github.com/}{GitHub}/\href{https://about.gitlab.com/}{GitLab})
    \item Algoritmos, diagramas de flujo y diseño de pseudocódigo.
\end{enumerate}

\section{Estructura básica de la programación (4 semanas)} \label{Sec: Estructura básica de la programación (4 semanas)}

\begin{enumerate}[label=\arabic*.]

    \item Operadores aritméticos y tipos de datos numéricos. (Precedencia y asociatividad)
    \item Sistemas numéricos de punto flotante y error numérico. (Épsilon de máquina y propagación de error)
    \item Tipos de datos de texto y arreglos. (Matrices, vectores, índices y subarreglos)
    \item Variables, constantes y funciones. (Manejo de memoria y recursividad)
    \item Operaciones lógicas y valores Booleanos. (Operadores lógicos y de comparación)
    \item Condicionales y control de flujo. (Declaraciones \texttt{if}, \texttt{else} y \texttt{elseif})
    \item Ciclos. (Ciclos iterativos \texttt{while}, \texttt{for}, y ciclos recursivos)
    \item Estructura de la programación modular. (Bibliotecas)
\end{enumerate}

\section{Representaciones visuales (3 semanas)} \label{Sec: Representaciones visuales (3 semanas)}

\begin{enumerate}[label=\arabic*.]

    \item Gráficación de funciones y animaciones con \href{https://docs.juliaplots.org/latest/}{Plots}.
    \item Visualización de datos con \href{https://docs.juliaplots.org/latest/}{Plots}.
    \item Visualización de sistemas de ecuaciones diferenciales ordinarias con \href{https://docs.juliaplots.org/latest/}{Plots}.
    \item Manipulación de imágenes digitales con \href{https://juliaimages.org/latest/}{JuliaImages}.
\end{enumerate}

\section{Cómputo científico (4 semanas)} \label{Sec: Cómputo científico: construcción de pseudocódigo e implementación en código (4 semanas)}

\begin{enumerate}[label=\arabic*.]

    \item Métodos numéricos. (Estabilidad y convergencia).
    \item Solución de sistemas lineales de ecuaciones algebráicas. (Método de eliminación Gaussiana)
    \item Aproximación de raíces. (Método de Newton)
    \item Solución de ecuaciones diferenciales ordinarias. (Método de Euler)
    \item Caminante aleatorio.
\end{enumerate}

\section{Introducción a otros lenguajes de programación (2 semanas)} \label{Sec: Introducción a otros lenguajes de programación (2 semana)}

\begin{enumerate}[label=\arabic*.]

    \item Introducción a Python. (Sintáxis, tipos de datos básicos y funciones con cantidades variables de parámetros)

    \item Introducción a Programación Orientada a Objetos. (Clases y objetos, parámetros y atributos, herencia y polimorfismo)

    \item Introducción a \href{https://www.manim.community/}{Manim}.
\end{enumerate}

\section*{Bibliografía recomandada para el curso} \label{Sec: Bibliografía}

\begin{enumerate}

    \item \href{https://benlauwens.github.io/ThinkJulia.jl/latest/book.html}{Lauwens y Downey, \emph{Think Julia: How to Think Like a Computer Scientist} (2019).}

    \item Burden, Faires y Burden, \emph{Numerical Analysis} (2016).

    \item Cormen, Leiserson, Rivest y Stein, \emph{Introduction to Algorithms} (2009).

    \item Cairó, \emph{Metodología de la programacióñ: Algoritmos, diagramas de flujo y programas} (2003).
\end{enumerate}

\section*{Bibliografía complementaria} \label{Sec: Bibliografía complementaria}

\begin{enumerate}

    \item \href{https://docs.julialang.org/en/v1/}{Documentación de Julia}.

    \item \href{https://docs.jupyter.org/en/latest/}{Documentación de Jupyter}.

    \item Material del curso \href{https://computationalthinking.mit.edu/Spring21/}{``Introduction to Computational Thinking''} del MIT.

    \item Blum y Bresnahan - \emph{Linux Command Line and Shell Scripting Bible} (2015).
\end{enumerate}


\end{document}
